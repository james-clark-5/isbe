% Briefly outline our attempt at solving the problem, and why it should be better than the solutions that have preceded it
This paper presents three contributions that advance the current body of knowledge in the analysis of curvilinear image structures.

First, we provide a detailed comparison of commonly used image filter banks in the context of their suitability for extracting the local image information pertaining to curvilinear structure. In doing so, we describe the desirable properties of a suitable filter bank, therefore enabling us to make a principled choice that balances the richness of the extracted information against efficiency (both computational and storage). This analysis leads us to apply a previously unexploited image filter -- the dual-tree complex wavelet transform (\dtcwt{}) -- to the task of curvilinear feature analysis. The \dtcwt{} benefits from specific properties, such as efficiency and phase information, that address the shortcomings of existing popular image filters (\sref{s:filtering}).

Second, we investigate and compare supervised learning approaches that, after training on a set of (input feature vector, target label) example pairs, turn a previously unseen feature vector of filter responses at a given location into an estimated output value. Specifically, we include the Random Forest statistical method -- not investigated in any similar study of which we are aware -- to predict outputs, and compare this to techniques that have been previously used. For line detection (\ie~image segmentation), we show that a carefully selected filter bank, coupled with a modern learning method that is capable of dealing with non-linear data, produces results that match or exceed the state-of-the-art without relying on application-specific assumptions to boost performance. For orientation, we show how to formulate the learning problem such that angle wraparound is dealt with correctly, and in doing so produce\comment{ significantly} better estimates than existing methods that compute orientation analytically. In addition, we show how predicting output responses via machine learning allows us to estimate the error associated with the predictions, which we propose may in itself be of use in further processing.

Third, we explore the similarities and differences between these different approaches, and present empirical evidence of which work best in practice for applications including retinography~(\sref{s:retinography}), mammography~(\sref{s:mammography}) and [another?].

In all cases, we back up theoretical claims for filter performance with thorough experimental validation on both synthetic and real data. 

%Finally, if for a given image class we know of additional properties that will be useful to predict and that can be reliably labelled on our training data (for example, a further classification of structure in aerial images to roads and rivers) these can easily be added to our protocol. 
