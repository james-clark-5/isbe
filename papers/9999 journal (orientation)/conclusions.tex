In conclusion, we see that filters based only on odd filters perform poorly near the centre of a curvilinear structures whereas even filters such as the second derivatives are much better, though a feature set comprising the responses to both odd and even filters offers further advantages. 
There is also potential to approximate the second derivative filter responses with Haar-like features if efficiency is a key concern, though it is less clear how to combine these responses to give a unique solution. 
Of the filters we tested, we found that the \dtcwt{} gave the best results regardless of the regressor used, though was significantly more computationally expensive. Of the regressors we tested, Random Forests performed best and we have provided some insight as to why alternatives (\eg~linear regression) perform less well. 
Combining suitable filters with Random Forests produces vessel segmentation that matches the state of the art without application specific post-processing as used in rival methods (and that we would expect to improve results further). 
Moreover, we have shown that regressing orientation estimates using similar machinery is more accurate than relying on analytical estimations. 
Though demonstrated on retinograms, our methods are generally applicable to linear structures in any images where ground truth is available.
Most promisingly, we note the larger improvement in orientation estimation for particularly challenging structures such as thin, low-contrast vessels. As a further advantage of regressing with random forests, we propose that both the predicted orientation \emph{and} its associated magnitude may be useful features in further processing.
\comment{We must, however, take care when building regressors for orientation prediction in order to ensure that angles wrap around the circle correctly.}