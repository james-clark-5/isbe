%Breast screening programmes using x-ray mammography have been deployed widely to detect early signs of cancer. The use of computer-aided detection systems to support breast radiology experts is also widespread. In mammograms, the projection of a complex network of vessels, ducts and connective tissue in the breast results in images that are rich in linear structures at a variety of scales and orientations. Mammographic signs of cancer include: the presence of a radio-opaque mass, often with associated radiating curvilinear structures (spicules); the presence of microcalcifications (clusters of highly radio-opaque 'dots'); and the presence of abnormal or abnormally organised curvilinear structures (architectural distortion - AD). Fig 3(a) shows an approximately 4 x 4 cm region extracted from a mammogram, centred on a malignant tumour, and demonstrates a central mass and radiating spicules, along with normal curvilinear structures (vessels, ducts, glandular tissue etc). The signs of abnormality often appear in conjunction, but when only AD is present, detection of the abnormality is extremely difficult. It has been estimated that approximately a third of all cancers missed during mammographic screening present as AD [3], and it has been reported that computer-aided detection systems do not yet detect AD with adequate sensitivity or specificity [4].

