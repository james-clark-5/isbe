Our proposal is to develop a new technology to measure the small blood vessels (capillaries)
of the finger nailfold. Nailfold capilaries are almost always abnormal in the multisystem disease
scleroderma �systemic sclerosis', and their early detection by naifold capillaroscopy nailfold
examination under a microscope) helps clinicians distinguish �primary� (benign) Raynaud�s
phenomenon (episodic discoloration of the fingers in cold weather) and scleroderma
associated Raynaud�s. Patients with scleroderma can have very severe finger problems with
scarring, ulcers and even gangrene, so early diagnosis is important.

With the new technology, researchers will be able not only to detect if capillaries are normal or
abnormal, but also to quantify the degree of abnormality and measure change over time, or in
response to drug treatment. Current treatments for sc eroderma associated Raynaud�s are
unsatisfactory, because our present inabi ity to measure the disease process accuratey has
hampered cinica trias of new treatments The capi aroscopy system we have deve oped in
Manchester measures abnorma ity and tracks change over time, but needs major refinement
to make it �use friend y�, fast, and comp ete y automatic Once deve oped, the new techno ogy
wi be used by c inicians across different countries to research sc eroderma and other
diseases such as diabetes in which abnorma ities of sma b ood vesse s occur

Patients with SSc spectrum disorders suffer major pain and disabi ity from digita vascu ar
disease, and the major cha enges are to (a) deve op effective therapies and (b) diagnose
disease ear y, a owing eary inten/ention with these therapies before deve opment of
irreversibe tissue injury The ack of re iabe outcome measures has pagued cinica trias of
SSc re ated digita vascu ar disease f nai fo d capi aroscopy becomes va idated as a re iab e
and feasibe biomarker for digita vascu ar disease, then this wi faciitate cinica trias by
reducing tria size, a owing the SSc research community to eva uate a arger number of drugs
over the next 5 years than woud be possibe using current outcome measures aone The
different objectives are important, because (in turn):
� To be feasib e for c inica and research use, images must be easy and quick to acquire
� Fu y automated measurement wi remove subjectivity and save time
- Measuring ve ocity wi a ow b ood fow as we as capi ary structure to be
characterised, providing comp imentary data which may be more sensitive to short term
change (e g when assessing acute drug responses to different vasoactive therapies)
� The new system wi be fu y va idated to achieve scientific and c inica credibi ity
� Fu integration with cinica information systems wi a ow data storage as part of the
e ectronic patient record, faci itating routine c inica use and providing an inva uab e repository
for future ong term, mu ticentre co aborative studies inc uding observationa studies of disease
progression and of treatment response

Athough a re ativey uncommon disease, SSc is associated with major morbidity and morta ity,
often in young peope Preva ence in the adut popu ation is reported as 250 per mi ion [1] ts
c inica features resu t from a combination of microvascu ar abnorma ity, eading to ischaemic injury,
and fibrosis Athough interna organ invo vement is responsibe for the high morta ity of SSc, much
of the morbidity re ates to the digits (especia y the fingers but aso the toes): Raynaud's
phenomenon (episodic co our change of the fingers, usua y in response to co d) is the commonest
presenting feature n patients with SSc, Raynaud�s phenomenonldigita ischaemia can progress to
irreversib e tissue injury with painfu u ceration, scarring, and gangrene (Figure 1)

n patients attending Sa ford Roya Hospita , in the order of 15% of patients had one or more digita
amputations [2] Treatment of S50 re ated Raynaud's phenomenon is often ineffective, with itte
evidence base [3] C inica trias of SSc re ated Raynaud�s phenomenon/digita ischaemia have so
far been thwarted by the ack of re iab e measures of outcome: digita u ceration is often the primary
end point, yet a recent study found that there was very poor agreement between c inicians with an
interest in SSc in the definition of a digita u cer [4] The net resut is that phase & trias of
treatments for digita vascu ar disease are difficut to mount as they require a ong fo ow up and
arge numbers of patients Thus biomarkers for digita vascu ar disease in patients with SSc
spectrum disorders are bady needed Robust outcome measures of microvascu ar structure and
function wou d faci itate phase studies of new ines of therapy and acce erate phase tria s We
be ieve that recent advances in naifod capi aroscopy mean that this technique, with further
deve opment as proposed in this app ication, cou d provide the ong awaited re iab e biomarker for
SSc associated microvascu ar disease Advances in capi aroscopy (inc uding those proposed here)
shoud improve eary diagnosis of SSc spectrum disorders, a owing trias to be conducted (and
effective treatments, once deve oped, commenced) in patients with eary disease, who have not yet
progressed to irreversibe tissue injury Once deve oped, the new automated system coud be
app ied in other diseases which, athough not associated with the structural microvascular
abnorma ities of SSc, are characterised by microvascu ar dysfunction: as discussed be ow, the
potentia of naifo d capi aroscopy to measure functional change in diabetes and other
microvascu ar diseases has yet to be exp oited

Scleroderma affects around 1 in 900 people in the United States (approximately 300\,000 total)