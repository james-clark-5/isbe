\label{s:filtering_first_derivs_derivation}
%
Defining $\G_{(1)}(\theta)$ as the first-order derivative filter at angle $\theta$,
%
\begin{align}
G_{(1)}(\theta)
	&= 	\frac{\partial G}{\partial r} \\
	&= 	\frac{\partial G}{\partial x}\frac{\partial x}{\partial r} +
			\frac{\partial G}{\partial y}\frac{\partial y}{\partial r} \\
	&= 	\Gx \cos(\theta) + \Gy \sin(\theta)
\label{e:dG}
\end{align}

\noindent where $\Gx = G_{(1)}(0^\circ)$ and $\Gy = G_{(1)}(90^\circ)$. The response to this filter is
%
\begin{align}
R_{(1)}(\theta)
	&= 	\G_{(1)}(\theta) \ast I \\
	&=	(\Gx \cos(\theta) + \Gy \sin(\theta)) \ast I \\
	&=	(\Gx \ast I) \cos(\theta) + (\Gy \ast I) \sin(\theta) \\
	&=	\Ix \cos(\theta) + \Iy \sin(\theta)
\label{e:R1}
\end{align}

\noindent where $\Ix$ and $\Iy$ are the responses to $\Gx$ and $\Gy$, respectively. That the response at any $\theta$ is a linear sum of the response at two orientations is the property of \emph{steerability} that makes computing the response at any angle efficient. Differentiating and equating to zero gives
%
\begin{align}
\frac{d}{d\theta}R_{(1)}
	&= -\Ix \sin(\theta) + \Iy \cos(\theta) = 0 \\
\Rightarrow \tan(\theta)
	&= \frac{\Iy}{\Ix}.
\label{e:t1}
\end{align}
